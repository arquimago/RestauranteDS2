%%% Pacotes utilizados %%%

%% Codificação e formatação básica do LaTeX
% Suporte para português (hifenação e caracteres especiais)
\usepackage[english,brazilian]{babel}
% Codificação do arquivo
\usepackage[utf8]{inputenc}
% Mapear caracteres especiais no PDF
\usepackage{cmap}
% Codificação da fonte
\usepackage[T1]{fontenc}
% Essencial para colocar funções e outros símbolos matemáticos
\usepackage{amsmath,amssymb,amsfonts,textcomp}
% Unidades de medida
\usepackage{siunitx}
% Identa o primeiro parágrafo
\usepackage{indentfirst}

%% Elementos Gráficos
% Para incluir figuras (pacote extendido)
\usepackage[]{graphicx}
% Package para figuras em containers
\usepackage{float}
% Suporte a cores
\usepackage{color}
% Criar figura dividida em subfiguras
\usepackage{subfig}
% Legenda da figura
\captionsetup[subfigure]{style=default, margin=0pt, parskip=0pt, hangindent=0pt, indention=0pt, singlelinecheck=true, labelformat=parens, labelsep=space}
% Customizar as legendas de figuras e tabelas
\usepackage{caption}
% Criar ambientes com 2 ou mais colunas
\usepackage{multicol}
% Criar figuras compostas de duas ou mais figuras
%\usepackage{subfig}

%% Tabelas
% Elementos extras para formatação de tabelas
\usepackage{array}
% Tabelas com qualidade de publicação
\usepackage{booktabs}
% Para criar tabelas maiores que uma página
\usepackage{longtable}
% adicionar tabelas e figuras como landscape
\usepackage{lscape}
% girar página no pdf
\usepackage{pdflscape}
% rotacionar imagens e tabelas
\usepackage{rotating}
% package para criar tabelas
\usepackage{tabulary}
% escrever algoritmos
\usepackage[portuguese,onelanguage]{algorithm2e}


%% Lista de Abreviações
% Cria lista de abreviações
\usepackage[notintoc,portuguese]{nomencl}
\makenomenclature


%% Notas de rodapé
\usepackage{footnote}
% Notas criadas nas tabelas ficam no fim das tabelas
\makesavenoteenv{tabular}


%% Links dinâmicos
% Suporte para hipertexto, links para referências e figuras
\usepackage{hyperref}
% Configurações dos links e metatags do PDF a ser gerado
\hypersetup{colorlinks=true, linkcolor=blue, citecolor=blue, filecolor=blue, urlcolor=blue}
% Conta o número de páginas
\usepackage{lastpage}


%% Referências bibliográficas e afins
\usepackage{natbib}
% Adicionar bibliografia, índice e conteúdo na Tabela de conteúdo (não inclui lista de tabelas e figuras no índice)
\usepackage[nottoc,notlof,notlot]{tocbibind}


%% Pontuação e unidades
% Posicionar inteligentemente a vírgula como separador decimal
\usepackage{icomma}
% Formatar as unidades com as distâncias corretas
\usepackage[tight]{units}


%% Layout
% Para definir espaçamento entre as linhas
\usepackage{setspace}
% Espaçamento do texto para o frame
\setlength{\fboxsep}{1em}
% Package para editar margens
\usepackage{anysize}
% margem esquerda, paginas impares
\setlength{\oddsidemargin}{0cm}  
% margem esquerda páginas pares (igual impares)
\setlength{\evensidemargin}{0cm} 
% Espaçamento 1,5
%\onehalfspacing
% Customizar enumerates
\usepackage{enumitem}

%% Inserir código-fonte em linguagens de programação
% Pacote utilizado
\usepackage{listings}
% Configurações do código
\lstset{
	language          = Java,                 % the language of the code
	%backgroundcolor   = \color{white},       % choose the background color; you must add \usepackage{color} or \usepackage{xcolor}
	basicstyle        = \tiny\ttfamily,       % the size of the fonts that are used for the code
	%breakatwhitespace = false,               % sets if automatic breaks should only happen at whitespace
	breaklines        = true,                 % sets automatic line breaking
	%caption           = \lstname,
	captionpos        = b,                    % sets the caption-position to bottom
	commentstyle      = \color[rgb]{0,0.6,0}, % comment style
	%deletekeywords    = {...},               % if you want to delete keywords from the given language
	%escapeinside      = {\%*}{*)},           % if you want to add LaTeX within your code
	extendedchars     = true,                 % lets you use non-ASCII characters; for 8-bits encodings only, does not work with UTF-8
	frame             = single,	              % adds a frame around the code
	inputpath         = {../Restaurante/app/src/main/java/ds2/equipe1/restaurante},
	keepspaces        = true,                 % keeps spaces in text, useful for keeping indentation of code (possibly needs columns=flexible)
	keywordstyle      = \color{blue},         % keyword style
	%otherkeywords     = {*,...},             % if you want to add more keywords to the set
	numbers           = left,                 % where to put the line-numbers; possible values are (none, left, right)
	%numbersep         = 5pt,                 % how far the line-numbers are from the code
	%numberstyle       = \tiny\color{mygray}, % the style that is used for the line-numbers
	%rulecolor         = \color{black},       % if not set, the frame-color may be changed on line-breaks within not-black text
	%showspaces        = false,               % show spaces everywhere adding particular underscores; it overrides 'showstringspaces'
	%showstringspaces  = false,               % underline spaces within strings only
	%showtabs          = false,               % show tabs within strings adding particular underscores
	%stepnumber        = 2,                   % the step between two line-numbers. If it's 1, each line will be numbered
	stringstyle       = \color[rgb]{0.8,0,0}, % string literal style
	tabsize           = 4,	                  % sets default tabsize to 2 spaces
	title             = \lstname              % show the filename of files included with \lstinputlisting; also try caption instead of title 
}
% Altera a caption Listing para a palavra escolhida
\renewcommand{\lstlistingname}{Código}


%% Cabeçalho e rodapé
% Controlar os cabeçalhos e rodapés
\usepackage{fancyhdr}
% Usar os estilos do pacote fancyhdr
\pagestyle{fancy}
\fancypagestyle{plain}{\fancyhf{}}
% Limpar os campos do cabeçalho atual
\fancyhead{}
% Número da página no rodapé
\fancyfoot[C]{\thepage}
% Omitir linha de separação entre cabeçalho e conteúdo
\renewcommand{\headrulewidth}{0pt}
% Omitir linha de separação entre rodapé e conteúdo
\renewcommand{\footrulewidth}{0pt}
% Altura do cabeçalho (13.6pt)
\headheight 0pt

\usepackage{lscape}


%% Comandos customizados
\newcommand{\universidade}{Universidade Federal de Sergipe}
\newcommand{\centro}{Centro de Ciências Exatas e Tecnologia}

\newcommand{\departamento}{DEPARTAMENTO DE COMPUTAÇÃO - DCOMP}
\newcommand{\curso}{CIÊNCIA DA COMPUTAÇÃO}

\newcommand{\titulo}{Documentos Desenvolvimento de Software}

\newcommand{\professor}{Prof. Dr. MICHEL DOS SANTOS SOARES}

\newcommand{\alunoA}{ELTON MOREIRA CARVALHO -- 201310004602}
\newcommand{\alunoB}{FERNANDO MELO NASCIMENTO -- 201210009310}
\newcommand{\alunoC}{FERNANDO MESSIAS DOS SANTOS -- 201320001408}
\newcommand{\alunoD}{RODRIGO BENEDITO OTONI -- 201210009188}
\newcommand{\alunoE}{THALES FRANCISCO SOUZA SAMPAIO ALVES DOS SANTOS -- 201210012648}


\newcommand{\etal}{et~al.}
\newcommand{\ie}{i.~e.}
\newcommand{\eg}{e.~g.}

\newcommand{\useCase}[5]{
	\textbf{Nome:} #1.\par
	\textbf{Descrição:} #2.\par
	\textbf{Identificador:} #3.\par
	\textbf{Importância:} #4.\par
	\textbf{Ator Primário:} #5.\par
	\textbf{Fluxo Principal:}\par}
	
\newcommand{\useCasePreCondition}[6]{
	\textbf{Nome:} #1.\par
	\textbf{Descrição:} #2.\par
	\textbf{Identificador:} #3.\par
	\textbf{Importância:} #4.\par
	\textbf{Ator Primário:} #5.\par
	\textbf{Pré-condições:} #6.\par
	\textbf{Fluxo Principal:}\par}