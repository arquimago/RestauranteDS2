\section{Plano de Projeto}

Devido ao avanço tecnológico e a crescente necessidade de encontrar soluções mais aprimoradas para o gerenciamento de um negócio, empresas buscam cada vez mais por sistemas que auxiliem no controle e gestão de suas atividades.

Esse projeto tem como objetivo a criação de um sistema para gerenciamento de restaurantes, permitindo o controle de diversas funções como: estoque, atendimento e gestão de funcionários. Esse sistema será integrado em computadores juntamente com o uso de tecnologias móveis de forma a permitir uma maior eficiência tanto no atendimento ao cliente como no controle das funções do restaurante. Devido a essa necessidade, além de uma rede de computadores, é necessário também que o estabelecimento disponha de dispositivos móveis (tablets ou smartphones) para o uso nas atividades .

O prazo necessário para conclusão das etapas de desenvolvimento e implantação do projeto foi estimado em 132 dias úteis.

\subsection{Motivação}

As atividades realizadas em um restaurante produzem uma grande quantidade de informações como: pedidos de clientes, compras com fornecedores, contratação de funcionários e gerenciamento de estoque. Devido a isso, surgem diversas dificuldades como:

\begin{itemize}
\item Gerência de grande volume de papel oriundo da venda de refeições, da compra de itens com fornecedores e da administração de funcionários.
\item Lentidão no atendimento aos clientes, sendo esta uma das principais causas de cancelamento dos pedidos.
\item Erros no preparo de pedidos dada a má compreensão do que é anotado em comandas.
\item Erros de cálculo quando a conta de uma comanda é solicitada.
\end{itemize}
